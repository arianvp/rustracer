\documentclass{article}
\usepackage{hyperref}
\title{Assignment 1: Raytracing framework}
\author{Arian van Putten - 4133935}
\begin{document}
\maketitle

\section{Implemented features}
\subsection{Basic features}
All basic features have been implemented.

\subsection{Extra features}

\subsubsection{Vulkan and Rust}
I opted to implement my ray tracer in the Programming Language Rust. A modern replacement for C++.

Rendering is performed by Vulkan. Vulkan is the successor of both OpenGL
  and OpenCL. It gives very low level access to the GPU, both to the rasterizer
    and to the compute pipeline.  Currently, the rasterizer is used to blit an
    image on the screen.  During startup, the user can select between either
    using the vulkan Compute pipeline to execute the raytracer, to to run the
    raytracer on the CPU.

    Vulkan drivers don't speak GLSL, but speak SPIR-V, a custom portable
    bytecode for GPUs.  Shaders can be written in any language, and can then be
    compiled to SPIR-V using LLVM and the Vulkan toolchain. In theory, this
    means I could write my compute shaders in Rust, but I haven't gotten it
    working yet.  So now I currently use GLSL-Compute.

    GPU Ray tracing is not yet implemented, and for now simply renders a
    Mandelbrot, to show that the Compute pipeline is working. I'm planning to
    port the current CPU tracer to Vulkan Compute in exercise 3, and the
    framework to do so is already there.

    The most interesting feature of Vulkan is that there is no concept of a
    rendering thread, and all operations are non-blocking and return so called
    "futures" that allow us to asynchronously subscribe to when the GPU is done
    with your command buffer.  This allows us to easily batch operations that
    are sent to the GPU, but also do operations in parallel to the GPU with
    ease without using a dedicated worker thread. I will use this in the future
    to rebuild the BVH in parallel during rendering.


\subsubsection{Realistic refraction}
Realistic refraction has been implemented using Schlicks approximation of the Frensel equations
and Beer's law for absorbance.:

\subsubsection{Triangle meshes and OBJ loader}
Triangle meshes are supported. I also support loading in triangle meshes from OBJ files.

Currently, there are some limitations:
\begin{itemize}
  \item Only one mesh per OBJ is supported, though the format ofifically supports multiple meshes per OBJ
  \item Material properties are not loaded from the OBJ, but set globally per
    mesh. However, changing this should not be too much work.
  \item Normal interpolation is currently not yet performed for the triangles, but could be added easily
  \item Texture mapping is not implemented
\end{itemize}

\subsubsection{Multiple light sources}
Multiple light sources are supported. Though they are all point lights.

\subsubsection{Multithreading}
On the CPU, work is divided evenly over all cores. For this I use the \texttt{scoped\_threadpool} library.

\subsubsection{Performance tuning}
Using the \texttt{prof} sampling profiling tool, it was found that almost all time of the
program (72 percent) was spent in the triangle intersection code, so I decided to tackle only only that code.

First of all, we start with a classic high level optimisation. Early out. I enabled \emph{backface culling} for diffuse materials. This improves the performance a lot!
Without it, most viewing angles give us a seconds per frame of around $1.60$ seconds.  But with back-face culling
enabled, seconds per frame drops down to $0.7$ seconds at certain viewing angles.

Second of all, I have rewritten the triangle intersect code using SIMD intrinsics. (See \texttt{src/tracer/primitive/triangle.rs, src/vec/mod.rs})

This
brings down the average seconds per frame to $1.02$ seconds, whilst the best case is still around $0.7$ seconds.
However, some small artifacts appear in the triangles. Either this is due to a bug, or due to the fact that
the \emph{Reciprocal} instruction is only an approximation. I have not yet had time to debug this, but would be interesting to look at in asssignment two.

A final profiling round shows that the handwritten SIMD code is now dominated by converting between
SIMD \texttt{f32x4} and Rust's \texttt{Vector3}.  I use \texttt{Vector3} in all of the other code, so
I only convert to \texttt{f32x4} when doing the triangle intersect. But apparently, this has quite some overhead.
However, due to time constraints, I was not yet able to move fully to \texttt{f32x4} for all vector arithmetic so I will leave this as a promising next step for Assignment two.




\subsection{Build instructions}
So this is a bit complicated. I'm working with a lot of new technologies, so building this can be non-trivial.
The following things are needed, at least:
\begin{itemize}
    \item A Driver that supports Vulkan.  Intel and NVIDIA have good support
    \item Vulkan SDK
    \item a GLSL to SPIR-V Compiler (glslValidator has one built in)
    \item CMake
    \item Rust nightly (The unstable version of rust) for SIMD instructions
    \item Probably Linux (I have not tested this code on Windows at all. It probably doesn't build on Windows)
\end{itemize}

Finally, we build the binary but make sure to enable SIMD intrinsics \texttt{
  RUSTFLAGS="-C target-cpu=native" cargo run --release}

Once all these tools are installed,  a simple \texttt{cargo run --release} in the project directory should build
all the dependencies and fire up the ray tracer.


\section{Sources}
As I already told Jacco Bikker, last year I worked together with Renier Maas on his path tracer that he built
for Advanced Graphics. I worked  on parallelisation and safe multi-threading for the Conepts of Program Design course.
Source code of that tracer can be found here: \url{https://github.com/arianvp/PathTracer}.

The free camera is based off the free camera I have had lying around since
Bachelor year two, which was then used in a GPU Path Tracer assignment in the
Concurrency Course, and is almost the same as the one used in Reinier's Path
Tracer (As I implemented it there). It has all kinds of advanced movement, but
also support for depth of field (if used in a path tracer), and changing the
field of view.

Multi-threading is very similar as that previous path tracer. Chunks of memory are simply
submitted to a threadpool using the \texttt{scoped\_threadpool} library.

Finally I'd like to note that \url{https://www.scratchapixel.com/} is a
\emph{great} resource for learning about computer graphics.  I referenced it a
lot to first refresh my knowledge from the bachelor graphics course on basic
shading.  But also for building sphere intersections, and basic refractions.

Furthemore the following libraries have been used:
\begin{itemize}
  \item \emph{tobj} A port of tinyobjectloader for Rust
  \item \emph{vulkano} Vulkan API for rust
  \item \emph{cgmath} Library implementing vector operations
  \item \emph{scoped\_threadpool} Multi-threading library for rust
\end{itemize}


\end{document}
