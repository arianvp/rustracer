\documentclass{article}
\title{Assignment 2: BVH}
\author{Arian van Putten - 4133935}
\begin{document}
\maketitle

\section{Implemented features}

Originally, my plan was to improve the  existing library
in rust named. However
I ended up reimplementing the BVH from scratch, as there
were lots of changes needed for the packed representation . However
it is still a fork, and I might be able to turn it into a Proper PR with some work.

Surprisingly, at least on the provided micro benchmarks, the cache aligend
traversal has not a lot of effect on the traversal performance. Once
could argue whether it is even worth it.

However, ordered traversal and a stack-based approach did affect the
performance a lot.


\begin{itemize}
  \item Stack based traversal (no recursion)
  \item Cache-line aligned allocator for Rust (It didn't have one)
  \item Packed BVH representation (nodes in 32 bytes)
  \item Ordered traversal
  \item Fast AABB-Ray intersections using SIMD
  \item Fast Binned construction with SAH along
  \item Ranged ray-packet traversal (using morton-curve ordering)
    has been implemented in the BVH  
    but is not used yet in the raytracer itself ,as I have not yet
    implemented sending packets of rays from the camera due to to time constraints :(
  \item Render Sponza scene with 2 frames per second.
  \item Render Mario 8 Wario Mountain on 4 to 10 frames per second.
\end{itemize}



